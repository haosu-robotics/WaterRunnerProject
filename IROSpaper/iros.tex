\documentclass[letterpaper, 10 pt, conference]{ieeeconf}  % Comment this line out
                                                          % if you need a4paper

\IEEEoverridecommandlockouts                              % This command is only
                                                          % needed if you want to
                                                          % use the \thanks command
\overrideIEEEmargins
% See the \addtolength command later in the file to balance the column lengths
% on the last page of the document



% The following packages can be found on http:\\www.ctan.org
\usepackage{graphics} % for pdf, bitmapped graphics files
%\usepackage{epsfig} % for postscript graphics files
%\usepackage{mathptmx} % assumes new font selection scheme installed
%\usepackage{times} % assumes new font selection scheme installed
\usepackage{amsmath} % assumes amsmath package installed
%\usepackage{amssymb}  % assumes amsmath package installed
\usepackage{hyperref}
\usepackage{todonotes}

\title{\LARGE \bf
Roll and Pitch Motion Control for Legged Locomotion \\ on the Water Surface
}

\author{Nitish Thatte$^{1}$, Mahdi Khoramshahi$^{2}$, and Metin Sitti$^{3}$% <-this % stops a space
%\thanks{*This work was not supported by any organization}% <-this % stops a space
\thanks{$^{1}$N. Thatte is with the NanoRobotics Laboratory, Robotics Institute, Carnegie Mellon University, 5000 Forbes Ave, Pittsburgh, PA 15213, USA 
	{\tt\small nitisht at andrew.cmu.edu}}%
\thanks{$^{2}$ M. Khoramshahi is with the NanoRobotics Laboratory, Robotics Institute, Carnegie Mellon University, 5000 Forbes Ave, Pittsburgh, PA 15213, USA 
	{\tt\small m80.khoramshahi at gmail.com}}
\thanks{$^{3}$ M. Sitti is the director of the NanoRobotics Laboratory, Department of Mechanical Engineering and Robotics Institute, Carnegie Mellon University, 5000 Forbes Ave, Pittsburgh, PA 15213, USA. Tel: 412-268-3632
	{\tt\small sitti at cmu.edu}}
}

\begin{document}

\maketitle
\thispagestyle{empty}
\pagestyle{empty}

%%%%%%%%%%%%%%%%%%%%%%%%%%%%%%%%%%%%%%%%%%%%%%%%%%%%%%%%%%%%%%%%%%%%%%%%%%%%%%%%
\begin{abstract}
	Previous work on a biologically-inspired quadrupedal water runner robot demonstrated successful stabilization of robot pitch motions through the use of an active tail. This approach however, has two drawbacks: it necessitates the integration of an additional actuator and  it cannot control roll motions. This paper proposes a novel approach for controlling roll and pitch motions for legged robots locomoting on the water surface. First, A simple model for legged water surface dynamics is developed. A control parameter, duty factor, which governs the timing of the plunge and retract phases of the foot trajectory is introduced, and a roll and pitch motion control strategy that uses this parameter is proposed. This control strategy is then applied to a detailed model of the actual water runner robot, and simulated experimental results are reported.
\end{abstract}

%%%%%%%%%%%%%%%%%%%%%%%%%%%%%%%%%%%%%%%%%%%%%%%%%%%%%%%%%%%%%%%%%%%%%%%%%%%%%%%%
\section{INTRODUCTION}

\section{LEGGED LOCOMOTION MODEL}

Before we can design a controller to stabilize the roll and pitch motion of the robot, we must first understand the dynamics of the system. Previous works, \cite{glasheen1996hydrodynamic}, \cite{floyd2008design}, \cite{hsieh2004running}, have considered the coupled vertical and horizontal kinematics and dynamics of the basilisk lizard and a water runner robot inspired by this animal. Both the lizard and the robot produce lift by slapping and stroking their feet through the water to generate lift via drag forces. The high speed of the vertical foot entry creates and air cavity through which the foot is retracted, thereby greatly reducing downwards drag forces during this phase.

The above studies provide detailed descriptions of forces generated during water surface locomotion, but their complexity can limit intuition and preclude the development of analytical expressions for generated force and robot state.  Therefore, to simplify and focus our analysis on the dynamics required to understand pitch and roll motions, we will only consider vertical dynamics in the following analysis. 


Galsheen and McMahon \cite{glasheen1996vertical} found that the time-varying vertical forces exerted by water during impact of disks follow the relation

\begin{equation}
	F(t) = - C_D^* \left[\frac{1}{2} S \rho \dot{y}(t) |\dot{y}(t) | + S \rho g y(t) \right]
\end{equation}

\noindent where $F(t)$ is the time varying drag force, $C_D^* \approx 0.703$ is the drag coefficient, $\rho$ is the density of water, $g$ is the acceleration due to gravity, $S = \pi r_{eff}^2$ is the effective circular area of the disk, and $y(t)$ and $\dot{y}(t)$ are the time-varying vertical position and velocity of the disk measured in a coordinate system where positive $y$ opposes the gravity vector. 

\section{CONCLUSIONS}

%\addtolength{\textheight}{-12cm}   % This command serves to balance the column lengths
                                  % on the last page of the document manually. It shortens
                                  % the textheight of the last page by a suitable amount.
                                  % This command does not take effect until the next page
                                  % so it should come on the page before the last. Make
                                  % sure that you do not shorten the textheight too much.

%%%%%%%%%%%%%%%%%%%%%%%%%%%%%%%%%%%%%%%%%%%%%%%%%%%%%%%%%%%%%%%%%%%%%%%%%%%%%%%%

\section*{APPENDIX}

Appendixes should appear before the acknowledgment.

\section*{ACKNOWLEDGMENT}

This material is based upon work supported by the National Science Foundation Graduate Research Fellowship under Grant No. (NSF grant number). Any opinion, findings, and conclusions or recommendations expressed in this material are those of the authors(s) and do not necessarily reflect the views of the National Science Foundation.

\bibliographystyle{IEEEtran}
\bibliography{iros}

\end{document}
