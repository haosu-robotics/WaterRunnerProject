\documentclass{article}

\usepackage{amsmath}
\usepackage{amssymb}
\usepackage{upgreek}
\usepackage{graphicx}
\usepackage[font={footnotesize,it}]{caption}
%\usepackage[margin=0.5in]{geometry} %remove comment after todonotes are dealt with
\usepackage[textwidth=1.5in]{todonotes}
\usepackage{parskip}

\begin{document}
\title{CPG formalization}
\author{Mahdi Khoramshahi}
\maketitle

\todo[inline]{I'll discuss the controller and motivation up here}
Equations for one node in the network.

\begin{align}\label{eq:one_node}
	\dot \phi_i &= \omega_{b} + \sum_{j \neq i} k_{ij} \sin \left( \phi_j - \phi_i - \tilde{\varphi}_{ij} \right ) + \zeta_i\\
	\Gamma_i    &= A \cos(\phi_i)
\end{align}

where $\phi_i$ is the phase of \textit{i}th node, $k_{ij}$ and $\tilde{\varphi}_{ij}$ are coupling gain and desired phase lag between $i$th and $j$th node in the network. $\zeta_i$ is the input force for the $i$th node, which is determined by the controller discussed in the previous section. Amplitude and frequency, $A$ and $\omega$ are equal for all nodes in the system. And finally, $\Gamma_i$ is the output of node $i$ and will be used as the desired trajectory for the $i$th motor.

In our network, $i$ can be any of four actuators on the robot, namely \textit{front-right}, \textit{front-left}, \textit{hind-right}, and  \textit{hind-left}, as shown in figure below, respectively nodes 1 to 4.

\begin{figure}[thpb]
	\centering
		\centering
		\includegraphics[scale = 0.95]{schema_cropped.pdf}
		\centering
		\caption{CPG network}
		\label{fig:CPG_Network}
\end{figure}

In the CPG network, $\varphi_{ij}$ determines the gait for robot. In order to ensure moments on the body are balanced we chose the the trot gait. Therefore, the phase lags in matrix form are:

\begin{equation}
	\tilde{\Phi} = [\tilde{\varphi}_{ij}]=\left[ \begin{array}{rrrr}
	0    & \pi & \pi  & 0   \\
	-\pi & 0   & 0    & \pi  \\
	-\pi & 0   & 0    & \pi   \\
	0    & -\pi&-\pi  & 0 
	\end{array} \right]
\end{equation}
	
This matrix is a skew-symmetric because of the phase lag property:

\begin{equation}
	\tilde{\varphi}_{ij} = - \tilde{\varphi}_{ji}
\end{equation}

As illustrated in Figure~\ref{fig:CPG_Network}, where dashed-grey line means coupling gain of zero, we are not using fully connected network. \todo{Can you explain why we are't using a fully connected network here} We are using the same coupling gains in the network. Therefore, the matrix representation of the coupling gains is:
 
\begin{equation}
	\upkappa = [k_{ij}]=\left[ \begin{array}{rrrr}
	0 & k & k & 0  \\
	k & 0 & 0 & k  \\
	k & 0 & 0 & k  \\
	0 & k & k & 0 
	\end{array} \right]
\end{equation}

The differential equations for our CPG-Network are as follows:

\begin{align}\label{eq:cpg_net}
	\dot\phi_1&=\omega_{b} + k\sin \left(\phi_2-\phi_1-\tilde{\varphi}_{12}\right)+k\sin\left(\phi_3-\phi_1-\tilde{\varphi}_{13}\right)+ \zeta_1\\
 	\dot\phi_2&=\omega_{b} + k\sin \left(\phi_1-\phi_2-\tilde{\varphi}_{21}\right)+k\sin\left(\phi_4-\phi_2-\tilde{\varphi}_{24}\right)+ \zeta_2\\
 	\dot\phi_3&=\omega_{b} + k\sin \left(\phi_1-\phi_3-\tilde{\varphi}_{31}\right)+k\sin\left(\phi_4-\phi_3-\tilde{\varphi}_{34}\right)+ \zeta_3\\
	\dot\phi_4&=\omega_{b} + k\sin \left(\phi_2-\phi_4-\tilde{\varphi}_{42}\right)+k\sin\left(\phi_3-\phi_4-\tilde{\varphi}_{43}\right)+ \zeta_4
\end{align}

Using the following linear transformation, $\varphi_{ij} = \phi_j - \phi_i - \tilde{\varphi}_{ij}$, and knowing that $\dot{\tilde{\varphi}}_{ij} = 0$ and $\varphi_{ij} = - \varphi_{ji}$, we arrive at the differential equations in phase lags coordination: \todo{Something seems wrong about the ending of this sentence. I'm not sure what its trying to say. Do you mean phase lag coordinates?}

\begin{align}\label{eq:cpg_net2}
	\dot\phi_1&=\omega_{b} + k\sin \left(\varphi_{12}\right)+k\sin\left(\varphi_{13}\right)+ \zeta_1\\
 	\dot\phi_2&=\omega_{b} + k\sin \left(\varphi_{21}\right)+k\sin\left(\varphi_{24}\right)+ \zeta_2\\
 	\dot\phi_3&=\omega_{b} + k\sin \left(\varphi_{31}\right)+k\sin\left(\varphi_{34}\right)+ \zeta_3\\
	\dot\phi_4&=\omega_{b} + k\sin \left(\varphi_{42}\right)+k\sin\left(\varphi_{43}\right)+ \zeta_4
\end{align}

\begin{align}
	\dot\varphi_{12}&= -2k\sin \left(\varphi_{12}\right)-k\sin\left(\varphi_{13}\right)+ k\sin\left(\varphi_{24}\right) + \zeta_2-\zeta_1 \label{eq:cpg_net3_1}\\
	\dot\varphi_{13}&= -2k\sin \left(\varphi_{13}\right)-k\sin\left(\varphi_{12}\right)+ k\sin\left(\varphi_{34}\right) + \zeta_3-\zeta_1\\
	\dot\varphi_{24}&= -2k\sin \left(\varphi_{24}\right)+k\sin\left(\varphi_{12}\right)- k\sin\left(\varphi_{34}\right) + \zeta_4-\zeta_2\\
	\dot\varphi_{34}&= -2k\sin \left(\varphi_{34}\right)+k\sin\left(\varphi_{13}\right)- k\sin\left(\varphi_{24}\right) + \zeta_4-\zeta_3 \label{eq:cpg_net3_4}
\end{align}

Using a small angle approximation, we can express the above equations as the following linear autonomous system:

\begin{equation}
  	\dot{\Phi} = \frac{d}{dt} \Phi =\frac{d}{dt}
  	\left[\begin{array}{c}\varphi_{12} \\ \varphi_{13} \\ \varphi_{24} \\ \varphi_{34}\end{array}\right]
  	=k\left[ \begin{array}{rrrr}
	-2 & -1 &  1 &  0   \\
	-1 & -2 &  0 &  1  \\
	 1 &  0 & -2 & -1   \\
	 0 &  1 & -1 & -2 
	\end{array} \right] \Phi = \hat{A}\Phi.
\end{equation}
	
The eigenvalues of $\hat{A}$ are $-4k$, $-2k$ , $-2k$, and $0$. We expect one zero eigenvalue-value due to the loop in our CPG network, which creates the following linear relationship between phase lags

\begin{equation}
	\varphi_{12} - \varphi_{13} + \varphi_{24} - \varphi_{34} = 0
\end{equation}

Negative eigenvalues show that are our autonomous (unforced) system is locally, asymptotically stable and will converge to our desired phase lags. However, CPG inputs, $\zeta_i$, will still affect phase lag convergence. As we see from equations~\ref{eq:cpg_net3_1} to~\ref{eq:cpg_net3_4}, phase lags are sensitive to mutual difference of $\zeta_i$. Therefore, we should design the closed-loop system such that we minimize the following costs:

\begin{equation}
	\min \zeta_2-\zeta_1, \ \zeta_3-\zeta_1, \ \zeta_4-\zeta_2, \ \zeta_4-\zeta_3 
    \label{eq:one_node1}
\end{equation}

This minimization is equivalent to the following minimization

\begin{equation}	
	\min \sigma_{\zeta} = \sum\limits_{i=1}^4 \left( \zeta_i - \bar{\zeta} \right)^2 \textrm{, where } \bar{\zeta} = \frac{1}{4}\sum\limits_{i=1}^4\zeta_i
	\label{eq:one_node2}
\end{equation}

Therefore, to better achieve convergence to our desired phase lags, it is very important to reduce variance of CPG inputs ($\zeta_i$). We take advantage of the dynamical nullspace and define this constraint (low variance input) as a task for the controller.

\section*{Input Scheduling}

In order to control downward and upward motion separately, we split each CPG node input ($\zeta_i$) in two parts as follows:

\begin{align}\label{eq:one_node3}
	\dot \phi_i &= \omega_{b} + \sum_{j \neq i} k_{ij} \sin \left( \phi_j - \phi_i - \tilde{\varphi}_{ij} \right ) + \zeta_i^+ + \zeta_i^-\\
	\Gamma_i    &= A \cos(\phi_i)
\end{align}

These inputs have the property $\zeta_i^+  \zeta_i^- = 0$, which means they cannot be active simultaneously. We can distinguish upward and downward motions by checking the sign of $\sin(\phi)$ as shown in Figure~\ref{fig:phase_plot}.

\begin{figure}[thpb]
	\centering
		\centering
		\includegraphics[scale = 0.7]{phase_plot_cropped.pdf}
		\centering
		\caption{phase}
		\label{fig:phase_plot}
\end{figure}

CPG forces ($\omega^+$ , $\omega^-$) are provided by our controller. Thus, in order to build our CPG force ($\zeta_i$), we will use: 

\begin{align}\label{eq:spliting}
	\zeta_i^+ &= \frac{1}{2}[1+ \mathrm{sign}(\sin\phi)] \omega^+\\
	\zeta_i^- &= \frac{1}{2}[1- \mathrm{sign}(\sin\phi)] \omega^-\\
	\zeta_i &= \zeta_i^+ + \zeta_i^-
\end{align}



\end{document}
