The Basilisk Lizard's striking ability to sustain highly dynamic legged locomotion on a range of surfaces from hard-ground to water is a remarkable feat \cite{glasheen1996hydrodynamic}. Most legged robots would have difficulty emulating this animal's ability to robustly locomote on yielding or deforming surfaces. Therefore, to explore the dynamics of legged locomotion in this regime, we are studying the design of a bio-inspired water-running robot. Analyzing water-running dynamics may also help us gain insight into mobility on other yielding surfaces, such as granular media and mud. It is crucial that we develop locomotion models for these surfaces as robots continue to venture out of the laboratory and into the real world.

Previous works,~\cite{glasheen1996hydrodynamic},~\cite{floyd2008design},~\cite{hsieh2004running}, have considered the coupled vertical and horizontal kinematics and dynamics of the basilisk lizard. These works find that the Basilisk Lizard produces lift and thrust by cycling its feet through the water in three distinct phases: slap, stroke, and recovery. In the slap and stroke phases, the lizard generates vertical and horizontal forces primarily via fluid drag on the foot. Then, during the retraction phase, the lizard withdraws its foot within the air cavity produced during the previous phases, thereby greatly reducing overall downwards forces. 

Based on this locomotion strategy a Park et al. designed and tested a bio-inspired water running robot. The robot could generate sufficient lift forces, but was unstable in roll and pitch \cite{park2010roll}. Undesired pitching motion was remedied by adding a tail that generates a balancing moment via fluid drag forces as the robot moves forward. However, this work proposed no method for controlling the roll or the height of the robot. Therefore, in this paper, we seek to model the dynamics of the water runner robot so that we can develop a controller to stabilize these degrees of freedom. 
