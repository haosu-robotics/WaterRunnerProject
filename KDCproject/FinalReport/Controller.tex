There are many possible techniques we can use to control the linear dynamical system given by equation~\ref{eq:rob_dyn}. We choose to use a virtual model combined with an inverse dynamics approach so that we can directly solve for the four leg frequencies that produce desired accelerations. Because our dynamics matrix $B$ is underdetermined, we can then use the nullspace of the dynamics matrix to consider other objectives and perform task prioritization.

We specify our desired behavior using a virtual model with a spring, damper, and integrator between the current and desired output:

\begin{align}
    \ddot{\vec{y}} &= K_p (\vec{y}_{des} - \vec{y}) + K_i \int_0^t (\vec{y}_{des} - \vec{y}) d\tau \notag \\
                         &\quad + K_d \frac{d}{dt} (\vec{y}_{des} - \vec{y}) \notag \\
                         &= K_p \vec{e} + K_i \int_0^t \vec{e} d \tau + K_d \frac{d}{dt} \vec{e} \label{eq:PID}
\end{align}

Plugging equation~\ref{eq:PID} into equation~\ref{eq:rob_dyn} and solving for $\vec{\omega}$ gives our control law:

\begin{align}
    \vec{\omega} &= B^{\dagger} \left[ M K_p \vec{e} + K_i \int_0^t \vec{e} d \tau + K_d \frac{d}{dt} \vec{e} - A \vec{y} - G \right] \notag \\
                 &\quad + (\mathbbm{1} - B^{\dagger} B) \vec{\omega}_0 \label{eq:control}
\end{align}

\noindent Where $B^{\dagger}$ is the psuedo-inverse of our dynamics matrix and $\mathbbm{1}$ is the identity matrix. The term $(\mathbbm{1} - B^{\dagger} B) \vec{\omega}_0$ projects $\vec{\omega}_0$ into the nullspace of our dynamics matrix. Therefore, the control inputs obtained by this control law will be as close to $\vec{\omega}_0$ as possible while providing our desired accelerations. As a result, we can use $\vec{\omega}_0$ to shape the control input to meet other desired objectives.

\begin{comment}
In our tests, we used a mix of two averaging schemes to specify $\omega_0$ at time step $t+1$ based on the leg velocities at the previous timestep $t$ The first scheme is as follows:

\begin{equation}
    \begin{bmatrix} \omega_1^L \\ \omega_1^R \\ \omega_2^L \\ \omega_2^R \end{bmatrix}_{t+1} = \begin{bmatrix} (\omega_1^L + \omega_2^L)/2 \\(\omega_1^R + \omega_2^R)/2 \\ (\omega_1^L + \omega_2^L)/2 \\ (\omega_1^R + \omega_2^R)/2 \end{bmatrix}_{t}
\end{equation}

\noindent This scheme should encourage upwards and downwards velocities that are close to each other, which may be beneficial for reducing motor accelerations and wear. The second scheme persues the opposite goal,

\begin{equation}
    \begin{bmatrix} \omega_1^L \\ \omega_1^R \\ \omega_2^L \\ \omega_2^R \end{bmatrix}_{t+1} = \begin{bmatrix} (\omega_1^L + \omega_1^R)/2 \\(\omega_1^L + \omega_1^R)/2 \\ (\omega_2^L + \omega_2^R)/2 \\ (\omega_2^L + \omega_2^R)/2 \end{bmatrix}_{t}
\end{equation}

\noindent and should encourage reliance on large differences between downwards and upwards velocities to produce force. This approach may be desirable in order to keep the legs in a certain gait, as it will prefer that the average frequency of the legs remains constant and equal between all legs of the robot. 
\end{comment}
