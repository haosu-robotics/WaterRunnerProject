\documentclass[letterpaper,twocolumn]{article}

\usepackage[top=0.5in, bottom=1in, left=1in, right=1in]{geometry}
\usepackage{float}
\usepackage{graphicx}
\usepackage{caption}
\usepackage{subcaption}
\usepackage{amsmath}


\title{Roll and Pitch Motion Control for Legged Locomotion on the Water Surface}

\author{\small Nitish Thatte, Mahdi Khoramshahi, Metin Sitti \\
        \small Carnegie Mellon University \\
        \small \{nitisht, khoram\} at andrew.cmu.edu, metin at cmu.edu
}
\date{}
\begin{document}

\maketitle

\section{Motivation}
The Basilisk Lizard's striking ability to sustain legged locomotion on a range of surfaces from hard-ground to the water is a remarkable feat \cite{glasheen1996hydrodynamic}. Currently, most legged robots would have difficulty emulating this animal's ability to robustly locomote on yielding or deforming surfaces. Therefore, to explore the dynamics of legged locomotion in this regime, we are studying the design of a bio-inspired water-running robot. Analyzing water-running dynamics may also help us gain insight into mobility on other yielding surfaces, such as granular media and mud.


\section{State of the Art}
The mechanical design of the water-running robot has undergone several design iterations. The current design, which can be seen in \ref{fig:robot}, features four four-bar linkage legs producing  foot trajectories optimized for power consumption and lift \cite{floyd2008design}. Park et al analyzed the roll and pitch dynamics of this robot and proposed both active and passive tails for achieving stable robot pitch angles \cite{park2010roll}. However, addition of an active tail requires integration of another actuator, adding mass and complexity. Furthermore, no method for controlling roll angles has been proposed as of yet.


\section{Own Approach}
To control the roll and pitch motion of the robot we propose a novel, closed-loop CPG that modulates the velocity of each foot during the downwards and upwards phases of their trajectories, while simultaneously maintaining the phase relation between feet. This approach takes advantage of non-linear fluid drag forces to generate differing forces on each foot, thereby imparting moments on the robot. 
	
To deal with the hybrid and time-variant nature of the system we utilize a time averaging approach. We then find the force generated by an approximated the foot trajectory in order to yield an expression for the generated foot pad forces. Finally, we design a non-linear PID controller based on this force law, which attempts to minimize the error between the height of each corner of the robot and a desired height, thereby achieving the desired roll and pitch angles.

\section{Current Results}

\section{Best Possible Outcome}

\section*{Acknowledgment}
This material is based upon work supported by the National Science Foundation Graduate Research Fellowships. Any opinion, findings, and conclusions or recommendations expressed in this material are those of the authors(s) and do not necessarily reflect the views of the National Science Foundation.

\bibliographystyle{plain}
\bibliography{dynamic}

\end{document}
