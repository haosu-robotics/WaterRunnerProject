\documentclass{article}

\usepackage{amsmath}
\usepackage{amssymb}
\usepackage{graphicx}
\usepackage[margin=1in]{geometry} %remove comment after todonotes are dealt with

\title{Research Proposal}
\author{Nitish Thatte}

\begin{document}
\maketitle

\section*{Introduction}
The Basilisk Lizard's striking ability to sustain highly dynamic legged locomotion on a range of surfaces from hard-ground to water is a remarkable feat~\cite{glasheen1996a}. Currently, most legged robots cannot emulate this animal's ability to robustly locomote on yielding or deforming surfaces. Rather, many legged robot platforms have mechanisms and controllers designed assuming a rigid ground with inelastic collisions \cite{phantom}. 

On soft or yielding surfaces, this assumption can be problematic, as numerous studies have demonstrated that locomotion on soft and yielding surfaces, such as sand and water, is very sensitive to leg stiffness and kinematics~\cite{Ferris1998}, \cite{Li2009}, \cite{Park2009}. For example, when running on hard ground, a robot requires compliance in its leg, as an inelastic ground collision diverts energy towards acceleration of actuator rotors~\cite{Hurst2005}. In contrast, when running in a liquid environment, high stiffness is required, because forces are primarily generated via drag. Any deflection of the leg will simply reduce the projected area or the velocity of the foot, thereby resulting in a loss of lift~\cite{Park2009}. Furthermore, on a liquid surface, the primary task is to generate sufficient lift. This task requires fundamentally different leg kinematics than locomotion on a hard surface, for which the emphasis is generating efficient forward motion. Given the vastly disparate requirements for legged locomotion across a broad spectrum of terrains, designing a bio-inspired multi-terrain running robot will require special attention towards designing mechanisms that feature adjustable compliance and kinematics while respecting strict mass budgets. During the design of this robot, we wish to develop guidelines, principles, and prioritites for roboticists designing multi-terrain robots in general.

Assuming we can design adaptive mechanisms that can produce leg trajectories and stiffnesses suitable for a variety of terrains. We must still propose algorithms and controlers to enable this rich locomotion behavior. These algorithms will fall under three categories:

\begin{enumerate}
	\item Methods to determine the properties of the current surface.
	\item Procedures for tuning the mechanisms to the current terrain i.e.\ the surface transition procedure.
	\item Controllers to ensure stable motion of the robot on the given surface.
\end{enumerate}



\bibliographystyle{plain}
\bibliography{proposal}
\end{document}
